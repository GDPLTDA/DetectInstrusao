% ------------------------------------------------------------------------
% ------------------------------------------------------------------------
% abnTeX2: Modelo de Trabalho Academico (tese de doutorado, dissertacao de
% mestrado e trabalhos monograficos em geral) em conformidade com 
% ABNT NBR 14724:2011: Informacao e documentacao - Trabalhos academicos -
% Apresentacao
% ------------------------------------------------------------------------
% ------------------------------------------------------------------------

\documentclass[
	% -- opções da classe memoir --
	12pt,				% tamanho da fonte
	openright,			% capítulos começam em pág ímpar (insere página vazia caso preciso)
	%twoside,			% para impressão em verso e anverso. Oposto a oneside
	oneside,
	a4paper,			% tamanho do papel. 
	% -- opções da classe abntex2 --
	%chapter=TITLE,		% títulos de capítulos convertidos em letras maiúsculas
	%section=TITLE,		% títulos de seções convertidos em letras maiúsculas
	%subsection=TITLE,	% títulos de subseções convertidos em letras maiúsculas
	%subsubsection=TITLE,% títulos de subsubseções convertidos em letras maiúsculas
	% -- opções do pacote babel --
	english,			% idioma adicional para hifenização
	french,				% idioma adicional para hifenização
	spanish,			% idioma adicional para hifenização
	brazil				% o último idioma é o principal do documento
	]{abntex2}

% ---
% Pacotes básicos 
% ---
\usepackage{lmodern}			% Usa a fonte Latin Modern			
\usepackage[T1]{fontenc}		% Selecao de codigos de fonte.
\usepackage[utf8]{inputenc}		% Codificacao do documento (conversão automática dos acentos)
\usepackage{lastpage}			% Usado pela Ficha catalográfica
\usepackage{indentfirst}		% Indenta o primeiro parágrafo de cada seção.
\usepackage{color}				% Controle das cores
\usepackage{graphicx}			% Inclusão de gráficos
\usepackage{microtype} 			% para melhorias de justificação
\usepackage[font={small,it}]{caption}

% ---
		
% ---
% Pacotes adicionais, usados apenas no âmbito do Modelo Canônico do abnteX2
% ---
\usepackage{lipsum}				% para geração de dummy text
% ---

% ---
% Pacotes de citações
% ---
\usepackage[brazilian,hyperpageref]{backref}	 % Paginas com as citações na bibl
\usepackage[alf]{abntex2cite}	% Citações padrão ABNT

% define o caminho das imagens
\graphicspath{{Imagens/}}

% --- 
% CONFIGURAÇÕES DE PACOTES
% --- 

% ---
% Configurações do pacote backref
% Usado sem a opção hyperpageref de backref
\renewcommand{\backrefpagesname}{Citado na(s) página(s):~}
% Texto padrão antes do número das páginas
\renewcommand{\backref}{}
% Define os textos da citação
\renewcommand*{\backrefalt}[4]{
	\ifcase #1 %
		Nenhuma citação no texto.%
	\or
		Citado na página #2.%
	\else
		Citado #1 vezes nas páginas #2.%
	\fi}%
% ---

% ---
% Informações de dados para CAPA e FOLHA DE ROSTO
% ---
\titulo{Uma abordagem de redes neurais artificiais para sistemas de detecção e prevenção de instrusão}
\autor{Rodrigo Mendonça da Paixão \\ Lucas Teles Agostinho}
\local{São Paulo -- Brasil}
\data{2015}
\orientador{Eduardo Heredia}
%\coorientador{Nome Completo}
\instituicao{%
  Centro Universitário Senac
  \par
  Bacharelado em Ciência da Computação
}
\tipotrabalho{Monografia (Graduação)}
% O preambulo deve conter o tipo do trabalho, o objetivo, 
% o nome da instituição e a área de concentração 
\preambulo{Pré-monografia apresentada na disciplina Trabalho de Conclusão de Curso I, como parte dos requisitos para obtenção do título de Bacharel em Ciência da Computação.}
% ---

% ---
% Configurações de aparência do PDF final

% alterando o aspecto da cor azul
\definecolor{blue}{RGB}{41,5,195}

% informações do PDF
\makeatletter
\hypersetup{
     	%pagebackref=true,
		pdftitle={\@title}, 
		pdfauthor={\@author},
    	pdfsubject={\imprimirpreambulo},
	    pdfcreator={LaTeX with abnTeX2},
		pdfkeywords={abnt}{latex}{abntex}{abntex2}{trabalho acadêmico}, 
		colorlinks=true,       		% false: boxed links; true: colored links
    	linkcolor=blue,          	% color of internal links
    	citecolor=blue,        		% color of links to bibliography
    	filecolor=magenta,      		% color of file links
		urlcolor=blue,
		bookmarksdepth=4
}
\makeatother
% --- 

% --- 
% Espaçamentos entre linhas e parágrafos 
% --- 

% O tamanho do parágrafo é dado por:
\setlength{\parindent}{1.3cm}

% Controle do espaçamento entre um parágrafo e outro:
\setlength{\parskip}{0.2cm}  % tente também \onelineskip

% ---
% compila o indice
% ---
\makeindex
% ---

% ----
% Início do documento
% ----
\begin{document}

% Retira espaço extra obsoleto entre as frases.
\frenchspacing 

% ----------------------------------------------------------
% ELEMENTOS PRÉ-TEXTUAIS
% ----------------------------------------------------------
% \pretextual

% ---
% Capa
% ---
\imprimircapa
% ---

% ---
% Folha de rosto
% (o * indica que haverá a ficha bibliográfica)
% ---
\imprimirfolhaderosto%*
% ---

% ---
% RESUMOS
% ---

% resumo em português
\setlength{\absparsep}{18pt} % ajusta o espaçamento dos parágrafos do resumo
%\begin{resumo}
   
 %\textbf{Palavras-chaves}: IDS,Rede,Internet
%\end{resumo}

% ---
% inserir lista de ilustrações (figuras)
% ---
%\pdfbookmark[0]{\listfigurename}{lof}
%\listoffigures*
%\cleardoublepage
% ---

% ---
% inserir lista de tabelas
% ---
%\pdfbookmark[0]{\listtablename}{lot}
%\listoftables*
%\cleardoublepage
% ---

% ---
% inserir lista de abreviaturas e siglas
% ---
\begin{siglas}
  \item[IDS] Intrusion Detection System
  \item[IPS] Intrusion Prevention System
  \item[NIDS] Network Intrusion Detection System
  \item[RNA] Rede Neural Artificial
  \item[SVM] Suport Vector Machine
  \item[AI] Artificial Intelligence
  \item[ML] Machine Learning
  \item[PNN] Rede Neural Probabilística
  \item[MLP] Multi-layered Perceptron  
\end{siglas}
% ---

% ---
% inserir o sumario
% ---
\pdfbookmark[0]{\contentsname}{toc}
\tableofcontents*
\cleardoublepage
% ---

% ----------------------------------------------------------
% ELEMENTOS TEXTUAIS 
% ----------------------------------------------------------
\textual

% ----------------------------------------------------------
% Capitulo 1
% ----------------------------------------------------------
\chapter[Introdução]{Introdução}
%\addcontentsline{toc}{chapter}{Introdução}
% ----------------------------------------------------------

\section{Contexto}

Nos dias atuais pessoas e empresas estão cada vez mais dependentes do uso da internet para realizar suas tarefas. Com o aumento da utilização também temos um aumento de casos de incidentes sobre quebra de segurança. Segundo o CERT \cite{CERT} tivemos um crescimento de 197\% de incidentes no ano de 2014 relativo a 2013. A necessidade de se proteger contra estes
ataques acabou despertando interesse por ferramentas automatizadas para detectar ataques e analisar formas de aprimorar as técnicas atuais para tal.

As ferramentas de detecção de intrusão são chamadas de IDS (Intrusion Detection System), seu trabalho é monitorar as atividades e analisar os eventos em uma rede em busca de anomalias que sugiram uma invasão. Estes não costumam executar qualquer ação para impedir intrusões, sua principal função é alertar os administradores de sistemas que há uma possível violação de segurança, sendo desta forma uma ferramenta passiva.
Existem as ferramentas de prevenção de intrusão, que são conhecidas como IPS (Intrusion Prevention System) estas são ferramentas que assim como IDS analisam o trafego e os eventos de uma rede, porem reagem de forma a bloquear o acesso ou atividade maliciosa, senso assim uma ferramenta ativa.

Podemos classificar os IDS da seguinte forma.

Baseado em Host ou rede, onde o primeiro faz uso de arquivos de log para cada computador individualmente e o segundo captura pacotes que trafegam na rede para analisar seu conteúdo.

Online ou Offline, onde um é capaz de detectar e marcar um intruso enquanto a esta sendo realizada a intrusão, e o outro analisa registros após o evento ocorrer e indica que houve uma violação de segurança tinha ocorrido desde a última verificação, respectivamente.

Baseado em abuso ou anomalia, onde por anomalia o sistema identifica comportamento fora do padrão, e por abuso compara as atividades na rede com comportamentos de ataques já conhecidos.

A maioria dos métodos utilizados para detecção são baseados em inteligencia artificial (AI), entre as varias técnicas conhecidas de AI, a que tem tido melhores resultados e consequentemente mais usada é a de Redes Neurais Artificiais (RNA)\cite{Jake-Ryan}\cite{Stampar}.

As RNA são uma classe de algorítimos para aprendizado de maquina (AM), usada para realizar classificação de dados. A rede neural é treinada de forma a dar mais importância para as principais características de uma determinada instancia de um problema, para ajudar a classificar os dados que ainda estão por vir. 
RNAs tem sido utilizadas com sucesso na detecção de intrusão \cite{Zhang} \cite{Tong} \cite{Wonil}, porem elas necessitam de uma quantidade substancial de dados para realizar o treinamento, a partir desse treinamento que ela passara a ter capacidade de identificar os padrões para posteriormente receber os dados novos para classificação.


\section{Motivação}

Esforços realizados para proporcionar segurança em ambientes computacionais, tem como motivação o fato de existirem riscos que podem comprometer a integridade, confiabilidade e disponibilidade da informação. 
Esses riscos são avaliados de acordo com as chances do mesmo ocorrer e com os custos envolvidos para tratá-lo. Técnicas de defesa vêm sendo aprimoradas, porém ainda existem diversas limitações que as impedem de estarem efetivamente preparadas para o qualquer tipo de ataque \cite{CeC}, sendo assim necessário  soluções inovadoras para tratar os níveis de ameaças atuais e futuras. 
Este cenário é a principal motivação deste trabalho que consiste em propor, implementar e mensurar resultados de uma solução para treinamento de RNA para detecção e prevenção de intrusão.

\section{Justificativa}

Muitas técnicas de AI tem sido utilizadas para IDS/IPS, a mais utilizada e a RNA\cite{Stampar}, porem existem tipos de ataques que não são facilmente detectados, por ocorrerem com menor frequência, tendo poucas entradas para o treino da RNA\cite{CeC}, resultando em mais erros,  por esse motivo escolhemos trabalhar de forma a aprimorar seu resultados. 


\section{Objetivos}

O objetivo deste trabalho é propor uma forma de aprimorar o sistema de aprendizado de redes neurais artificiais para detecção e prevenção de intrusão. 
Para isso sera necessário, gerar uma base em um ambiente controlado para testes específicos, implementar uma solução de IPS/IDS que utilize RNA, implementar uma metodologia de treinamento, realizar o treino da RNA e por fim comparar resultados com outras técnicas de treinamento para RNA.


\section{Método de trabalho}

Utilizaremos a base de dados de trafego em rede KDD Cup 99, por ser uma das bases mais completas e amplamente utilizada nos testes de IDS, sera essencial para se realizar uma comparação consistente de resultados.

Para gerar nossa base mais especifica iremos monitorar um ambiente de rede durante um período de tempo, no qual serão realizados alguns ataques controlados periodicamente, deste sera gerado um log que usaremos no nosso sistema.

Desenvolver uma solução para analise dos pacotes e eventos de uma rede, esta sera desenvolvida em Go, utilizara RNA para classificar as atividades na rede.

Realizar treinamento em ambas as bases de dados, serão formas diversificadas de treinamento, sera feito uma comparação de acertos/erros e tempo necessário para treino.

Faremos uma comparação de desempenho e efetividade de nossa solução e algumas que temos hoje.

\section{Organização do trabalho}

Este trabalho esta dividido em trés partes.
Na próxima seção, sera apresentado o estado da arte, onde sera revisado a literatura sobre detecção e prevenção de intrusão utilizando RNA.

Logo apos teremos a proposta  de forma mais detalhada do que é pretendido realizar na próxima etapa do trabalho.

Por fim um cronograma de controle sobre como prosseguira a segunda etapa deste trabalho.


% ----------------------------------------------------------
% Capitulo 2
% ----------------------------------------------------------
\chapter[Revisão de Literatura]{Revisão de Literatura}
%\addcontentsline{toc}{chapter}{Revisão de Literatura}
% ----------------------------------------------------------

Um trabalho feito na Information Systems Security Bureau \cite{Stampar}, fez um comparativo de técnicas publicas entre os anos de 2010 e 2014 para a detecção de intrusão.

A pesquisa indica que existe um pequeno crescimento das técnicas de aprendizado de maquina (ML) e inteligencia artificial (AI) comparadas com outras técnicas que não são informadas, mostrando as técnicas como uma sub-área importante e com uma forte tendencia, concluindo ML contribui como a principal área da AI utilizada para a detecção de intrusão.

Outro comparativo foi com o numero de publicação entre 2010 e 2014 das diferentes algoritmos de inteligencia artificial e aprendizado de maquina como Redes Neurais Artificiais, Logica Fuzzy, Algoritmo Genético, Arvore de Decisões, etc. A pesquisa mostrar que algorítimos com base em redes neurais artificiais são os algoritmos de inteligencia artificial mais popular entre as publicações, mesmo com uma pequena queda em seu uso no ultimo ano. Algoritmo Genético, K-vizinhos mais próximos e arvore de decisão ficam muito próximos, quase empatando no segundo lugar.

O artigo conclui que inteligencia artificial desempenha um papel substancial no estudo de detecção de intrusão, redes neurais artificiais  são as mais populares, e que a pesquisa nessa área ainda é muito necessária, ha muitos resultados promissores nesses algorítimos,  especialmente em abordagens hibridas, nos quais se utiliza a combinação entre técnicas diferentes.


No trabalho de Marley  \cite{Marley}, é dada uma visão geral sobre Redes Neurais Artificiais(RNA), explicando que a expressão "rede neural" é a tentativa de representar a capacidade que cérebro humano possui de reconhecer, associar e generalizar padrões, sendo as principais áreas de atuação para a classificação de padrões e previsão.

A modelagem de uma rede neural consiste em três etapas: (1) Treinamento e Aprendizado, obtido pelo ambiente gerador dos dados, (2) Associação, obtido pelo reconhecimento de padrões distintos e (3) Generalização, relacionado a capacidade da rede de reconhecer com sucesso o ambiente que origina os dados, não propriamente os dados utilizados no treinamento.

O conhecimento é passado por um algoritmo de treinamento e aprendizado, este é transformado e armazenado nas conexões. 
O aprendizado é resultado de muitas apresentações de conjuntos de exemplos de treinamento.

O treinamento pode ser dado de duas maneiras: (1)batelada ou ciclos, onde a atualização somente acontece depois da apresentação de todos os pesos e (2) padrão a padrão ou incremental,onde a atualização é feita após a apresentação de cada novo padrão. O procedimento de treinamento pode ser classificado em dois tipos: supervisionado e não supervisionado.

Muitas abordagens de RNA tem sido implementadas e testadas para IDS, um sistema proposto por Ryan \cite{Ryan}, analisou  o comportamento de usuários individuais. O padrão de entrada foi então combinado com os perfis de usuário para identificar o utilizador (Um nó correspondente ao utilizador com um valor > 0,5 é atribuída a esse usuário). Um flag é gerado se nenhuma correspondência for encontrada,  este é considerado como uma anomalia. No entanto, isto exigia grande quantidade de dados para treinar a rede para cada utilizador. Dados insuficientes para um usuário pode levar à falsos positivos para o comportamento desse usuário na rede. O sistema teve uma taxa de detecção de anomalia de 96\% e 7\% de falsos alarmes.

Cannady propôs um modelo chamado Multi-layered Perceptron (MLP) \cite{Cannady}, utilizando o algorítimo de \textit{backpropagation} ao invés de criar perfis individuais. Foi necessário 26.13 horas para completar com aproximadamente 98\% de acerto no conjunto de dados de treinamento e 97,5\% no conjunto de dados de teste.

Devaraju e Ramakrishnan implementaram a  Rede Neural Probabilística (PNN) \cite{Devaraju} esta obteve desempenho melhor do que uma rede do tipo FeedForward e Radial Basis. No entanto, a PNN (precisão = 80,38\%) conseguiu ser apenas 0,02\% melhor do que a rede do tipo FeedForward (precisão = 80,4\%) nao sendo uma diferença significativa. A precisão do  Radial Basis é de 75,4\%.
Resultados mais baixos que as abordagens que vimos até agora.

No trabalho em conjunto do Departamento de Automação e Sistemas da Universidade Federal de Santa Catarina \cite{polvo1} e com a Pontifícia Universidade Católica do Paraná \cite{polvo2}, foi desenvolvido um sistema multi-camadas chamado de POLVO-IIDS, utilizando redes neurais de Kohonen, que classifica os dados de forma genérica, comportamentos considerados normais ou anômalos, para cada classe de ataque foi utilizada um algoritmo de inteligencia artificial do tipo Support Vector Machine (SVM) especializada na detecção da classe correspondente e tendo como saída a indicação de trafego normal ou atividade maliciosa.
A ideia de utilizar outra rede neural é para minimizar o numero de falsos positivos, com apenas um tipo de ataque para classificar, aumenta a precisão para identificar apenas duas categorias, trafego normal ou anomalia.

O valor de cada neurônio pode variar de 0 a 1, normalmente em redes neurais de Kohonen algum neurônio deve estar de 0,1 a 0,9, mas no POLVO-IIDS, foi utilizado de 0.2 a 0.8 para evitar erros de margem. 
O trabalho indica de teve uma melhora nos resultados obtidos relacionados a outras literaturas, mostrando que o modelo POLVO-IIDS é um modelo eficiente.

Shraddha Surana propôs um modelo \cite{Surana} utilizando clusterização Fuzzy e Redes neurais artificiais, onde o dataset de treinamento é dividido em N subsets, que são distribuídos em N redes neurais intermediarias, com mais uma camada de RNA que agrega os resultados das redes intermediarias para a partir dai realizar a classificação de novos dados. Esta abordagem conseguiu uma taxa de detecção de 81,6\%.



Um trabalho feito na Pontifícia Universidade Católica do Rio de Janeiro \cite{RenatoMaia}, demonstra um teste de desempenho para algumas diferentes entradas para o treinamento de Redes neurais artificiais.

Utilizando 4 redes e no máximo 4 saídas possíveis, é testado diferentes composições, de forma a encontrar uma ideal maneira de treinamento de uma rede neural, utilizando as 41 categorias disponíveis.
As 4 são categorizas como Normal ou entre 3 tipos de ataque DoS Smurf, Nepture ou Back.
A primeira rede utiliza 9 entradas e uma saídas, onde a saída pode ser -1 para normal e 1 para ataque.
A segunda rede utiliza todas as 41 entradas e uma saída, onde a saída pode ser -1 para normal e 1 para ataque.
A terceira utliza nove entradas e uma saida, onde a saída pode ser -1 para normal,0 para Neptune e 1 Smurf.
A quarta e ultima rede, usa 9 entradas e quatro saídas que são ornizadas em (-1 1 1 1) para Normal,(1 -1 1 1) para Neptune,(1 1 -1 1) para Back e (1 1 1 -1) para Smurf.

Utilizando a base de dados KDD Cup 1999 para o treinamento das redes neurais, os resultados foram bons e com baixa taxa de falsos positivos, as taxas de acertos foram acima de 90 por cento para todas as configurações testadas, a rede que teve um melhor resultado foi a terceira rede, tendo 97,5\% na sua taxa de acertos contra.

No trabalho feito na universidade de Waterloo no canada\cite{Chunlin}, foi apresentado duas técnicas de redes neurais baseados em hierarquia para IDS, hierarquia em series e hierarquia paralela, onde o objeto para a detecção de ataques de abuso e anomalia em tempo real sem a interrupção humana.

Usando dois pré-requisitos para redes neurais hierarquias. O primeiro, cada classificação individual deve acertar seu desempenho, caso contrario,o erro é enviado para os níveis acima, acumulando e influenciados os níveis mais baixos. A taxa de detecção e de falso positivo são os principais indicadores de desempenho. 

A segunda, as classificações,basicamente, podem ser dividas em grupos seguindo alguns critérios, cada grupo pode ser associado para seu própria classificação, então as classificações ou sua saída pode ser combinada em conjuntos.

O artigo conclui em seus resultados, uma ótima habilidade para detectar instruções conhecidas e desconhecidas com um curte período de tempo para o treinamento, uma alta taxa de detecção e um baixo índice de falsos positivos.
O IDS de hierarquia de series conseguiu monitorar em tempo real o trafego na rede, treinando automaticamente novas intrusos modificando sua estrutura para novas classificações. O IDS de hierarquia paralela, resolveu em partes, os problemas da hierarquia em series, sendo mais rápido para ser executado.


% ----------------------------------------------------------
% Capitulo 3
% ----------------------------------------------------------
\chapter[Proposta]{Proposta}
%\addcontentsline{toc}{chapter}{Metodologia}
% ----------------------------------------------------------

Será implementada na próxima etapa desse trabalho um modelo para Detecção e prevenção de intrusão fazendo uso de técnicas de inteligência artificial, especificamente redes neurais artificiais, utilizar suas características de reconhecimento de padrões e generalização para realizar a classificação dos eventos ocorridos em redes de computadores, classificando-os em normais ou intrusivos.

Sera desenvolvida uma aplicação baseada no modelo, ela ira monitorar os pacotes em uma rede de computadores do tipo IPv4, este ira classificar e prevenir atividades maliciosas.

A principio sera utilizada uma RNA do tipo \textit{feed-foward}, com três camadas apenas, uma de entrada, uma intermediaria e uma de saída. Os dados serão passados para a camada de entrada, processados pela rede e classificados como uma das cinco classes da camada de saída. Estas sendo Normal, DoS, U2R, R2L e Probe.  

O modelo terá foco no treinamento da RNA, a principio para aprender os pesos da rede em multicamadas sera usado o algorítimo de \textit{backpropagation} com a regra de atualização de peso do gradiente descendente.

Desse ponto serão mensurados alguns testes a partir da abordagem do POLVO-IIDS \cite{polvo1}, no qual usaremos uma rede para classificação, e quatro SVM para detecção de anomalias.

Outra abordagem é a de clusterização \cite{Surana}, separando o \textit{dataset} de treino em N \textit{subsets}, cada um sera utilizado para realizar o treinamento de uma rede neural, para no final outra rede  realizar a agregação dos resultados destas N redes.

 
Pretende-se analisar a viabilidade de aplicação destas abordagems, bem como
detalhar suas vantagens ou desvantagens em relação a métodos convencionais de
detecção de intrusão.


% ----------------------------------------------------------
% Capitulo 4
% ----------------------------------------------------------
\chapter[Cronograma]{Cronograma}
%\addcontentsline{toc}{chapter}{Expectativas}
% ---

Para modelar a proposta descrita, sera seguido um cronograma semanal, este ira descrever semana a semana as tarefas que devem ser realizadas de forma a se concluir o trabalho.


\vspace{.25cm}
\begin{center}
	\begin{tabular}{ |c| p{13.5cm} | }
		\hline
		\textbf{Semana} & \textbf{Atividade} \\ \hline \hline
		$1^{\underline a}$ & Analise de trafego de pacotes em rede, criar aplicação em Go para ler e realizar log das operações na rede \\ \hline
		$2^{\underline a}$ & Criar base de dados de trafego em rede de forma controlada. Verificar como implementações de IDS atuais reagem a esses dados. / trabalhar na monografia \\ \hline
		$3^{\underline a}$ & Implementar na aplicação de log de trafego um sistema de redes neurais \\ \hline		
		$4^{\underline a}$ & Realizar treinamento da RNA da aplicação com os \textit{datsets} KDD'99 e o modelado para o projeto / trabalhar na monografia \\ \hline		
		$5^{\underline a}$ & Mensurar resultados da aplicação do modelo / trabalhar na monografia \\ \hline		
		$6^{\underline a}$ & Comparar com os resultados obtidos por outros IDS, e com resultados publicados de outros modelos baseados em RNA  \\ \hline		
		$7^{\underline a}$ & Analizar se for possivel como aprimorar os resultados do modelo \\ \hline		
		$8^{\underline a}$ & Implementar o modelo de POLVO-IIDS \\ \hline		
		$9^{\underline a}$ & Comparar com os resultados obtidos anteriormente / trabalhar na monografia \\ \hline		
		$10^{\underline a}$ & Implementar RNA clusterizado. \\ \hline		
		$11^{\underline a}$ & Comparar com os resultados obtidos anteriormente / trabalhar na monografia \\ \hline		
		$13^{\underline a}$ & Implementar modelo hibrido POLVO-IIDS Clusterizado \\ \hline		
		$14^{\underline a}$ & Comparar com os resultados obtidos anteriormente / trabalhar na monografia \\ \hline		
		$15^{\underline a}$ & Trabalhar na monografia - desenvolvimento \\ \hline		
		$16^{\underline a}$ & Trabalhar na monografia - resultados \\ \hline		
		$17^{\underline a}$ & Trabalhar na monografia - resultados \\ \hline		
		$18^{\underline a}$ & Apresentação dos resultados \\ \hline
	\end{tabular}
\end{center}

\vspace{1.25cm}


% ----------------------------------------------------------
% ELEMENTOS PÓS-TEXTUAIS
% ----------------------------------------------------------
\postextual
% ----------------------------------------------------------

% ----------------------------------------------------------
% Referências bibliográficas
% ----------------------------------------------------------
\bibliography{mono}


% Revisao Bibliografica
\end{document}
