% ------------------------------------------------------------------------
% ------------------------------------------------------------------------
% abnTeX2: Modelo de Trabalho Academico (tese de doutorado, dissertacao de
% mestrado e trabalhos monograficos em geral) em conformidade com 
% ABNT NBR 14724:2011: Informacao e documentacao - Trabalhos academicos -
% Apresentacao
% ------------------------------------------------------------------------
% ------------------------------------------------------------------------

\documentclass[
	% -- opções da classe memoir --
	12pt,				% tamanho da fonte
	openright,			% capítulos começam em pág ímpar (insere página vazia caso preciso)
	%twoside,			% para impressão em verso e anverso. Oposto a oneside
	oneside,
	a4paper,			% tamanho do papel. 
	% -- opções da classe abntex2 --
	%chapter=TITLE,		% títulos de capítulos convertidos em letras maiúsculas
	%section=TITLE,		% títulos de seções convertidos em letras maiúsculas
	%subsection=TITLE,	% títulos de subseções convertidos em letras maiúsculas
	%subsubsection=TITLE,% títulos de subsubseções convertidos em letras maiúsculas
	% -- opções do pacote babel --
	english,			% idioma adicional para hifenização
	french,				% idioma adicional para hifenização
	spanish,			% idioma adicional para hifenização
	brazil				% o último idioma é o principal do documento
	]{abntex2}

% ---
% Pacotes básicos 
% ---
\usepackage{lmodern}			% Usa a fonte Latin Modern			
\usepackage[T1]{fontenc}		% Selecao de codigos de fonte.
\usepackage[utf8]{inputenc}		% Codificacao do documento (conversão automática dos acentos)
\usepackage{lastpage}			% Usado pela Ficha catalográfica
\usepackage{indentfirst}		% Indenta o primeiro parágrafo de cada seção.
\usepackage{color}				% Controle das cores
\usepackage{graphicx}			% Inclusão de gráficos
\usepackage{microtype} 			% para melhorias de justificação
\usepackage[font={small,it}]{caption}

% ---
		
% ---
% Pacotes adicionais, usados apenas no âmbito do Modelo Canônico do abnteX2
% ---
\usepackage{lipsum}				% para geração de dummy text
% ---

% ---
% Pacotes de citações
% ---
\usepackage[brazilian,hyperpageref]{backref}	 % Paginas com as citações na bibl
\usepackage[alf]{abntex2cite}	% Citações padrão ABNT

% define o caminho das imagens
\graphicspath{{Imagens/}}

% --- 
% CONFIGURAÇÕES DE PACOTES
% --- 

% ---
% Configurações do pacote backref
% Usado sem a opção hyperpageref de backref
\renewcommand{\backrefpagesname}{Citado na(s) página(s):~}
% Texto padrão antes do número das páginas
\renewcommand{\backref}{}
% Define os textos da citação
\renewcommand*{\backrefalt}[4]{
	\ifcase #1 %
		Nenhuma citação no texto.%
	\or
		Citado na página #2.%
	\else
		Citado #1 vezes nas páginas #2.%
	\fi}%
% ---

% ---
% Informações de dados para CAPA e FOLHA DE ROSTO
% ---
\titulo{Título Provisório da Monografia de\\ Trabalho de Conclusão de Curso}
\autor{Rodrigo Mendonça da Paixão \\ Lucas Teles Agostinho}
\local{São Paulo -- Brasil}
\data{2015}
\orientador{Eduardo Heredia}
%\coorientador{Nome Completo}
\instituicao{%
  Centro Universitário Senac
  \par
  Bacharelado em Ciência da Computação
}
\tipotrabalho{Monografia (Graduação)}
% O preambulo deve conter o tipo do trabalho, o objetivo, 
% o nome da instituição e a área de concentração 
\preambulo{Pré-monografia apresentada na disciplina Trabalho de Conclusão de Curso I, como parte dos requisitos para obtenção do título de Bacharel em Ciência da Computação.}
% ---

% ---
% Configurações de aparência do PDF final

% alterando o aspecto da cor azul
\definecolor{blue}{RGB}{41,5,195}

% informações do PDF
\makeatletter
\hypersetup{
     	%pagebackref=true,
		pdftitle={\@title}, 
		pdfauthor={\@author},
    	pdfsubject={\imprimirpreambulo},
	    pdfcreator={LaTeX with abnTeX2},
		pdfkeywords={abnt}{latex}{abntex}{abntex2}{trabalho acadêmico}, 
		colorlinks=true,       		% false: boxed links; true: colored links
    	linkcolor=blue,          	% color of internal links
    	citecolor=blue,        		% color of links to bibliography
    	filecolor=magenta,      		% color of file links
		urlcolor=blue,
		bookmarksdepth=4
}
\makeatother
% --- 

% --- 
% Espaçamentos entre linhas e parágrafos 
% --- 

% O tamanho do parágrafo é dado por:
\setlength{\parindent}{1.3cm}

% Controle do espaçamento entre um parágrafo e outro:
\setlength{\parskip}{0.2cm}  % tente também \onelineskip

% ---
% compila o indice
% ---
\makeindex
% ---

% ----
% Início do documento
% ----
\begin{document}

% Retira espaço extra obsoleto entre as frases.
\frenchspacing 

% ----------------------------------------------------------
% ELEMENTOS PRÉ-TEXTUAIS
% ----------------------------------------------------------
% \pretextual

% ---
% Capa
% ---
\imprimircapa
% ---

% ---
% Folha de rosto
% (o * indica que haverá a ficha bibliográfica)
% ---
\imprimirfolhaderosto%*
% ---

% ---
% RESUMOS
% ---

% resumo em português
\setlength{\absparsep}{18pt} % ajusta o espaçamento dos parágrafos do resumo
\begin{resumo}
   
 \textbf{Palavras-chaves}: IDS,Rede,Internet
\end{resumo}

% ---
% inserir lista de ilustrações (figuras)
% ---
%\pdfbookmark[0]{\listfigurename}{lof}
%\listoffigures*
%\cleardoublepage
% ---

% ---
% inserir lista de tabelas
% ---
%\pdfbookmark[0]{\listtablename}{lot}
%\listoftables*
%\cleardoublepage
% ---

% ---
% inserir lista de abreviaturas e siglas
% ---
\begin{siglas}
  \item[IDS] Intrusion Detection System
  \item[IPS] Intrusion Prevention System
  \item[NIDS] Network Intrusion Detection System
  \item[IA] Inteligência Artificial
  \item[RNA] Rede Neural Artificial
  \item[IM] Inteligencia de maquina	
\end{siglas}
% ---

% ---
% inserir o sumario
% ---
\pdfbookmark[0]{\contentsname}{toc}
\tableofcontents*
\cleardoublepage
% ---

% ----------------------------------------------------------
% ELEMENTOS TEXTUAIS 
% ----------------------------------------------------------
\textual

% ----------------------------------------------------------
% Capitulo 1
% ----------------------------------------------------------
\chapter[Introdução]{Introdução}
%\addcontentsline{toc}{chapter}{Introdução}
% ----------------------------------------------------------

\section{Contexto}

Nos dias atuais pessoas e empresas estão cada vez mais dependentes do uso da internet para realizar suas tarefas. Com o aumento da utilização também temos um aumento de casos de incidentes sobre quebra de segurança. Segundo o CERT \cite{CERT} tivemos um crescimento de 197\% de incidentes no ano de 2014 relativo a 2013. A necessidade de se proteger contra estes
ataques acabou despertando interesse por ferramentas automatizadas para detectar ataques e analisar formas de aprimorar as técnicas atuais para tal.

As ferramentas de detecção de intrusão são chamadas de IDS (Intrusion Detection System), seu trabalho é monitorar as atividades e analisar os eventos em um rede em busca de anomalias que sugiram uma invasão. Estes não costumam executar qualquer ação para impedir intrusões, sua principal função é alertar os administradores de sistemas que há uma possível violação de segurança, sendo desta forma uma ferramenta passiva.
Existem as ferramentas de prevenção de intrusão, que são conhecidas como IPS (Intrusion Prevention System) estas são ferramentas que assim como IDS analisam o trafego e os eventos de uma rede, porem reagem de forma a bloquear o acesso ou atividade maliciosa, senso assim uma ferramenta ativa .

Podemos classificar os IDS da seguinte forma.

Baseado em Host ou rede, onde o primeiro faz uso de arquivos de log para cada computador individualmente e o segundo captura pacotes que trafegam na rede para analisar seu conteúdo.

Online ou Offline, onde um é capaz de detectar e marcar um intruso enquanto a esta sendo realizada a intrusão, e o outro analisa registros após o evento ocorrer e indica que houve uma violação de segurança tinha ocorrido desde a última verificação, respectivamente.

Baseado em abuso ou anomalia, onde por anomalia o sistema identifica comportamento fora do padrão, e por abuso compara as atividades na rede com comportamentos de ataques já conhecidos.

A maioria dos métodos utilizados para detecção são baseados em inteligencia artificial (IA), entre as varias técnicas conhecidas de IA, a que tem tido melhores resultados e consequentemente mais usada é a de Redes Neurais Artificiais (RNA)\cite{Jake-Ryan}\cite{Stampar}.

As RNA são uma classe de algorítimos para aprendizado de maquina (AM), usada para realizar classificação de dados. A rede neural é treinada de forma a dar mais importância para as principais características de uma determinada instancia de um problema, para ajudar a classificar os dados que ainda estão por vir. 
RNAs tem sido utilizadas com sucesso na detecção de intrusão \cite{Zhang} \cite{Tong} \cite{Wonil}, porem ela necessita de uma quantidade substancial de dados para realizar o treinamento, a partir desse treinamento que ela tera a capacidade de identificar os padrões para posteriormente receber os dados novos para classificação.


\section{Motivação}

Esforços realizados para proporcionar segurança em ambientes computacionais, tem como motivação o fato de existirem riscos que podem comprometer a integridade, confiabilidade e disponibilidade da informação. 
Esses riscos são avaliados de acordo com as chances do mesmo ocorrer e com os custos envolvidos para tratá-lo. Técnicas de defesa vêm sendo aprimoradas, porém ainda existem diversas limitações que as impedem de estarem efetivamente preparadas para o qualquer tipo de ataque \cite{CeC}, sendo assim necessário  soluções inovadoras para tratar os níveis de ameaças atuais e futuras. 
Este cenário é a principal motivação deste trabalho que consiste em propor, implementar e mensurar resultados de uma solução para treinamento de RNA para detecção e prevenção de intrusão.

\section{Justificativa}

Muitas técnicas de IA tem sido utilizadas para IDS/IPS, a mais utilizada e a RNA\cite{Stampar}, porem existem tipos de ataques que não são facilmente detectados, por ocorrerem com menor frequência, tendo poucas entradas para o treino da RNA\cite{CeC}, resultando em mais erros,  por esse motivo escolhemos trabalhar de forma a aprimorar seu resultados. 


\section{Objetivos}

O objetivo deste trabalho é propor uma forma de aprimorar o sistema de aprendizado de redes neurais artificiais para detecção e prevenção de intrusão. 
Para isso sera necessário, gerar uma base em um ambiente controlado para testes específicos, implementar uma solução de IPS/IDS que utilize RNA, implementar uma metodologia de treinamento, realizar o treino da RNA e por fim comparar resultados com outras técnicas de treinamento para RNA.


\section{Método de trabalho}

Utilizaremos a base de dados de trafego em rede KDD Cup 99, por ser uma das bases mais completas e amplamente utilizada nos testes de IDS, sera essencial para se realizar uma comparação consistente de resultados.

Para gerar nossa base mais especifica iremos monitorar um ambiente de rede durante um período, no qual serão realizados alguns ataques controlados periodicamente, sera gerado um log que usaremos na nosso sistema.

Desenvolver uma solução para analise dos pacotes e eventos de uma rede, esta sera desenvolvida em Go, utilizara RNA para classificar as atividades na rede.

Realizar treinamento em ambas as bases de dados, serão formas diversificadas de treinamento, sera feito uma comparação de acertos/erros e tempo necessário para treino.

Faremos uma comparação de desempenho e efetividade de nossa solução e algumas que temos hoje.

\section{Organização do trabalho}

Este trabalho esta dividido em trés partes.
Na próxima seção, sera apresentado o estado da arte, onde sera revisado a literatura sobre detecção e prevenção de intrusão utilizando RNA.

Logo apos teremos a proposta  de forma mais detalhada do que é pretendido realizar na próxima etapa do trabalho.

Por fim um cronograma de controle sobre como prosseguira a segunda etapa deste trabalho.


% ----------------------------------------------------------
% Capitulo 2
% ----------------------------------------------------------
\chapter[Revisão de Literatura]{Revisão de Literatura}
%\addcontentsline{toc}{chapter}{Revisão de Literatura}
% ----------------------------------------------------------

Um trabalho feito na Information Systems Security Bureau \cite{Stampar}, fez um comparativo de técnicas publicas entre os anos de 2010 e 2014 para a detecção de intrusão, a pesquisa indica que existe um pequeno crescimento das técnicas de aprendizado de maquina e inteligencia artificial comparadas com outras técnicas que não são informadas, mostrando as técnicas como uma sub-area importante e com uma forte tendencia, concluindo ML contribui como a principal área da AI e AI utilizada para a detecção de intrusão.Outro comparativo foi com o numero de publicação entre 2010 e 2014 das diferentes algoritmos de inteligencia artificial e aprendizado de maquina como Redes Neurais Artificiais,Logica Fuzzy, Algoritmo Genético, Arvore de Decisões,etc.A pesquisa mostrar que redes neurais artificiais são os algoritmos de inteligencia artificial mais populares entres as publicações,mas com uma pequena queda no ultimo ano. Algoritmo Genético, K-vizinhos mais próximos e arvore de decisão ficam muito próximos, quase empatando no segundo lugar. O texto conclui que inteligencia artificial desempenha um papel substancial no estudo de inteligencia artificial e redes neurais são as mais populares, mas pesquisa nessa área são necessárias por que ha muitos resultados promissores dos algoritmos e especialmente a area que utiliza a combinação entre eles.

Em uma trabalho em conjunto do Departamento de Automacao e Sistemas da Universidade Federal de Santa Catarina \cite{polvo1} e com a Pontifícia Universidade Católica do Paraná \cite{polvo2}, foi desenvolvido um sistema multi-camadas chamado de POLVO-IIDS, utilizando redes neurais de Kohonen, que classifica os dados de forma
genérica comportamentos considerados normais ou anômalo e para cada classe de ataque foi utilizada uma rede neural do tipo Support Vector Machine(SVM) especializada na detecção da classe correspondente e tendo como saída a indicação de trafego normal ou atividade maliciosa.
A ideia de utilizar outra rede neural é utilizada para minimizar o numero de falsos positivos, pois com apenas um tipo de ataque para classifica, aumenta a precisão para identificar apenas duas categorias trafego normal ou anomalia.O valor de cada neurônio pode variar de 0 a 1, normalmente em redes neurais de Kohonen algum neurônio deve estar de 0,1 a 0,9,mas no POLVO-IIDS, foi utilizado de 0,2 a 0,8 para evitar erro de algum possível ataque. O trabalho indica de teve uma melhora nos resultados obtidos relacionados a outras literaturas,mostrando que o modelo POLVO-IIDS é um modelo eficiente.

Um trabalho feito na Pontifícia Universidade Católica do Rio de Janeiro \cite{RenatoMaia}, demostra um teste de desempenho para algumas diferentes entradas para o treinamento de Redes neurais artificiais,
utilizando 4 redes, apenas uma utiliza as 41 categorias diferentes, as outras 3 utilizam apenas 9 categorias básicas do TCP-IP, usando um intervalo de -1 a 1. A primeira rede utilizando todas as 41 categorias e a segunda rede com apenas 9 categorias com apenas uma saída, identificando como 1 ataque ou -1 normal, a terceira com apenas uma saída e 9 categorias, indicando como -1 normal, 0 neptune ou 1 smurf e a ultima com 4 saídas e 9 categorias, indicando (-1 1 1 1) normal,(1 -1 1 1) neptune, (1 1 -1 1) back e (1 1 1 -1) smurf. As taxas de acertos foram acima de 90 por cento para todas as configurações testadas, a rede que teve um melhor resultado foi a terceira rede, tendo 97,5% na sua taxa de acertos.


Na Universidade Federal de Santa Maria \cite{Dalmazo},utilizando serie temporal que eh um modelo matemático para apresentar amostragens periódicas que apresentam dependência entre as amostras, classificando os sistemas de detecção de intrusão em três componentes fundamentais: fonte de informação, analise e resposta.A fonte de informação representada por um coletor associado a um host, rede ou segmento de rede.Analise sendo parte da SDI que verifica eventos derivados da fonte de informação onde eh determinada a indicação que um evento eh uma intrusão que esta ocorrendo ou ja ocorreu.O detector de intrusões baseado em series temporais teve como resultado preliminar uma demostração que a utilização de series temporais para a detecção de ataques, apresentam resultados satisfatórios para a detecção de um ataque e em pouco consumir de tempo de processamento.

% ----------------------------------------------------------
% Capitulo 3
% ----------------------------------------------------------
\chapter[Proposta]{Proposta}
%\addcontentsline{toc}{chapter}{Metodologia}
% ----------------------------------------------------------


% ----------------------------------------------------------
% Capitulo 4
% ----------------------------------------------------------
\chapter[Cronograma]{Cronograma}
%\addcontentsline{toc}{chapter}{Expectativas}
% ---



% ----------------------------------------------------------
% ELEMENTOS PÓS-TEXTUAIS
% ----------------------------------------------------------
\postextual
% ----------------------------------------------------------

% ----------------------------------------------------------
% Referências bibliográficas
% ----------------------------------------------------------
\bibliography{mono}


% Revisao Bibliografica
\end{document}
