% ------------------------------------------------------------------------
% ------------------------------------------------------------------------
% abnTeX2: Modelo de Trabalho Academico (tese de doutorado, dissertacao de
% mestrado e trabalhos monograficos em geral) em conformidade com 
% ABNT NBR 14724:2011: Informacao e documentacao - Trabalhos academicos -
% Apresentacao
% ------------------------------------------------------------------------
% ------------------------------------------------------------------------

\documentclass[
	% -- opções da classe memoir --
	12pt,				% tamanho da fonte
	openright,			% capítulos começam em pág ímpar (insere página vazia caso preciso)
	%twoside,			% para impressão em verso e anverso. Oposto a oneside
	oneside,
	a4paper,			% tamanho do papel. 
	% -- opções da classe abntex2 --
	%chapter=TITLE,		% títulos de capítulos convertidos em letras maiúsculas
	%section=TITLE,		% títulos de seções convertidos em letras maiúsculas
	%subsection=TITLE,	% títulos de subseções convertidos em letras maiúsculas
	%subsubsection=TITLE,% títulos de subsubseções convertidos em letras maiúsculas
	% -- opções do pacote babel --
	english,			% idioma adicional para hifenização
	french,				% idioma adicional para hifenização
	spanish,			% idioma adicional para hifenização
	brazil				% o último idioma é o principal do documento
	]{abntex2}

% ---
% Pacotes básicos 
% ---
\usepackage{lmodern}			% Usa a fonte Latin Modern			
\usepackage[T1]{fontenc}		% Selecao de codigos de fonte.
\usepackage[utf8]{inputenc}		% Codificacao do documento (conversão automática dos acentos)
\usepackage{lastpage}			% Usado pela Ficha catalográfica
\usepackage{indentfirst}		% Indenta o primeiro parágrafo de cada seção.
\usepackage{color}				% Controle das cores
\usepackage{graphicx}			% Inclusão de gráficos
\usepackage{microtype} 			% para melhorias de justificação
\usepackage[font={small,it}]{caption}
% ---
		
% ---
% Pacotes adicionais, usados apenas no âmbito do Modelo Canônico do abnteX2
% ---
\usepackage{lipsum}				% para geração de dummy text
% ---

% ---
% Pacotes de citações
% ---
\usepackage[brazilian,hyperpageref]{backref}	 % Paginas com as citações na bibl
\usepackage[alf]{abntex2cite}	% Citações padrão ABNT

% define o caminho das imagens
\graphicspath{{Imagens/}}

% --- 
% CONFIGURAÇÕES DE PACOTES
% --- 

% ---
% Configurações do pacote backref
% Usado sem a opção hyperpageref de backref
\renewcommand{\backrefpagesname}{Citado na(s) página(s):~}
% Texto padrão antes do número das páginas
\renewcommand{\backref}{}
% Define os textos da citação
\renewcommand*{\backrefalt}[4]{
	\ifcase #1 %
		Nenhuma citação no texto.%
	\or
		Citado na página #2.%
	\else
		Citado #1 vezes nas páginas #2.%
	\fi}%
% ---

% ---
% Informações de dados para CAPA e FOLHA DE ROSTO
% ---
\titulo{Título Provisório da Monografia de\\ Trabalho de Conclusão de Curso}
\autor{Rodrigo Mendonça da Paixão \\ Lucas Teles Agostinho}
\local{São Paulo -- Brasil}
\data{2015}
\orientador{Eduardo Heredia}
%\coorientador{Nome Completo}
\instituicao{%
  Centro Universitário Senac
  \par
  Bacharelado em Ciência da Computação
}
\tipotrabalho{Monografia (Graduação)}
% O preambulo deve conter o tipo do trabalho, o objetivo, 
% o nome da instituição e a área de concentração 
\preambulo{Pré-monografia apresentada na disciplina Trabalho de Conclusão de Curso I, como parte dos requisitos para obtenção do título de Bacharel em Ciência da Computação.}
% ---

% ---
% Configurações de aparência do PDF final

% alterando o aspecto da cor azul
\definecolor{blue}{RGB}{41,5,195}

% informações do PDF
\makeatletter
\hypersetup{
     	%pagebackref=true,
		pdftitle={\@title}, 
		pdfauthor={\@author},
    	pdfsubject={\imprimirpreambulo},
	    pdfcreator={LaTeX with abnTeX2},
		pdfkeywords={abnt}{latex}{abntex}{abntex2}{trabalho acadêmico}, 
		colorlinks=true,       		% false: boxed links; true: colored links
    	linkcolor=blue,          	% color of internal links
    	citecolor=blue,        		% color of links to bibliography
    	filecolor=magenta,      		% color of file links
		urlcolor=blue,
		bookmarksdepth=4
}
\makeatother
% --- 

% --- 
% Espaçamentos entre linhas e parágrafos 
% --- 

% O tamanho do parágrafo é dado por:
\setlength{\parindent}{1.3cm}

% Controle do espaçamento entre um parágrafo e outro:
\setlength{\parskip}{0.2cm}  % tente também \onelineskip

% ---
% compila o indice
% ---
\makeindex
% ---

% ----
% Início do documento
% ----
\begin{document}

% Retira espaço extra obsoleto entre as frases.
\frenchspacing 

% ----------------------------------------------------------
% ELEMENTOS PRÉ-TEXTUAIS
% ----------------------------------------------------------
% \pretextual

% ---
% Capa
% ---
\imprimircapa
% ---

% ---
% Folha de rosto
% (o * indica que haverá a ficha bibliográfica)
% ---
\imprimirfolhaderosto%*
% ---

% ---
% RESUMOS
% ---

% resumo em português
\setlength{\absparsep}{18pt} % ajusta o espaçamento dos parágrafos do resumo
\begin{resumo}
   
 \textbf{Palavras-chaves}: IDS,Rede,Internet
\end{resumo}

% ---
% inserir lista de ilustrações (figuras)
% ---
\pdfbookmark[0]{\listfigurename}{lof}
\listoffigures*
\cleardoublepage
% ---

% ---
% inserir lista de tabelas
% ---
\pdfbookmark[0]{\listtablename}{lot}
\listoftables*
\cleardoublepage
% ---

% ---
% inserir lista de abreviaturas e siglas
% ---
\begin{siglas}
  \item[IDS] Intrusion Detection System
  \item[IPS] Intrusion Prevention System
  \item[NIDS] Network Intrusion Detection System
  \item[IA] Inteligência Artificial
  \item[RNA] Rede Neural Artificial
  \item[IM] Inteligencia de maquina	
\end{siglas}
% ---

% ---
% inserir o sumario
% ---
\pdfbookmark[0]{\contentsname}{toc}
\tableofcontents*
\cleardoublepage
% ---

% ----------------------------------------------------------
% ELEMENTOS TEXTUAIS 
% ----------------------------------------------------------
\textual

% ----------------------------------------------------------
% Capitulo 1
% ----------------------------------------------------------
\chapter[Introdução]{Introdução}
%\addcontentsline{toc}{chapter}{Introdução}
% ----------------------------------------------------------

\section{Contexto}

Nos dias atuais pessoas e empresas estão cada vez mais dependentes do uso da internet para realizar suas tarefas. Com o aumento da utilização também temos um aumento de casos de incidentes de quebra de segurança. Segundo o CERT [1](Centro de Estudos, Resposta e Tratamento de Incidentes de Segurança no Brasil) tivemos um crescimento de 197\% de incidentes no ano de 2014 relativo a 2013. A necessidade de se proteger contra estes
ataques acabou despertando interesse por ferramentas automatizadas para detectar ataques e analisar formas de aprimorar as técnicas atuais para tal.

Chamamos as ferramentas de detecção de intrusão de IDS (Intrusion Detection System), seu trabalho é monitorar as atividades e analisar os eventos em um rede em busca de anomalias que sugiram uma invasão. Estes não costumam executar qualquer ação para impedir intrusões, sua principal função é alertar os administradores de sistemas que há uma possível violação de segurança, sendo assim uma ferramenta passiva.
Temos as ferramentas de prevenção de intrusão, que são conhecidas como IPS (Intrusion Prevention System) estas são ferramentas que assim como IDS analisa o trafego e os eventos de uma rede, porem reage de forma a bloquear o acesso ou atividade maliciosa, senso assim uma ferramenta ativa .

Podemos classificar os IDS da seguinte forma. 

Baseado em Host ou baseado em rede onde o primeiro faz uso de arquivos de log de cada computador individualmente e o segundo captura pacotes que trafegam na rede para analisar seu conteúdo.

Online ou Offline, onde um é capaz de detectar e marcar um intruso enquanto a esta sendo realizada a intrusão, e o outro analisa registros após o evento ocorrer e indica que houve uma violação de segurança tinha ocorrido desde a última verificação, respectivamente.

Baseado em abuso ou baseado em anomalia, onde por anomalia o sistema identifica comportamento fora do padrão, e por abuso compara as atividades na rede com comportamentos de ataques já conhecidos.

A maioria dos métodos utilizados para detecção são baseados em inteligencia artificial (IA), entre as varias técnicas conhecidas de IA, a que tem tido melhores resultados e consequente a mais usada é a de Redes Neurais Artificiais (RNA)[2][3].

As RNA são uma classe de algorítimos para aprendizado de maquina (AM), usada para realizar classificação de dados. A rede neural é treinada de forma a dar mais importância para as principais características de uma determinada instancia de um problema, para ajudar a classificar os dados que ainda estão por vir. 
RNAs tem sido utilizadas com sucesso na detecção de intrusão [4][5][6], sela necessita de uma quantidade substancial de dados para realizar o treinamento, a partir desse treinamento que ela tera a capacidade de identificar os padrões para depois receber os dados novos para classificação.


\section{Motivação}

Esforços realizados para proporcionar segurança em ambientes computacionais, tem como motivação o fato de existirem riscos que podem comprometer a integridade, confiabilidade e disponibilidade da informação. 
Esses riscos são avaliados de acordo com as chances do mesmo ocorrer e com os custos envolvidos para tratá-lo. Técnicas de defesa vêm sendo aprimoradas, porém ainda existem diversas limitações que as impedem de estarem efetivamente preparadas para o qualquer tipo de ataque[7], sendo assim necessário  soluções inovadoras para tratar os níveis de ameaças atuais e futuras. 
Este cenário é a principal motivação deste trabalho que consiste em propor, implementar e mensurar resultados de uma solução para treinamento de RNA para detecção e prevenção de intrusão.

\section{Justificativa}

Muitas tecnicas de IA tem sido utilizadas para IDS/IPS, a mais utilizada e a RNA[2], porem existem tipos de ataques que nao sao facilmente detectados, por ocorrerem com menor frequencia, tendo poucas entradas para o treino da RNA[7], resultando em mais erros,  por esse motivo escolhemos trabalhar de forma a aprimorar seu resultados.


\section{Objetivos}

Temos como objetivo propor uma forma de aprimorar o sistema de aprendizado de redes neurais artificiais para detecção e prevenção de intrusão.

Para isso sera necessário utilizar uma base de testes com grande volume de dados de trafego em rede.
Gerar uma base em um ambiente controlado para testes específicos.
Implementar uma solução de IPS/IDS que utilize RNA  para classificar os tipos de ataque.
Comparar resultados com outras técnicas de treinamento para RNA.


\section{Método de trabalho}

Utilizaremos a base de dados de trafego em rede KDD Cup 99, por ser uma das bases mais completas e amplamente utilizada nos testes de IDS, sera essencial para se realizar uma comparação consistente de resultados.

Para gerar nossa base mais especifica iremos monitorar um ambiente de rede durante um período, no qual serão realizados alguns ataques controlados periodicamente, sera gerado um log que usaremos na nosso sistema.

Desenvolver uma solução para analise dos pacotes e eventos de uma rede, esta sera desenvolvida em Go, utilizara RNA para classificar as atividades na rede.

Realizar treinamento em ambas as bases de dados, serão formas diversificadas de treinamento, sera feito uma comparação de acertos/erros e tempo necessário para treino.

Faremos uma comparação de desempenho e efetividade de nossa solução e algumas que temos hoje.

\section{Organização do trabalho}

Este trabalho esta dividido em trés partes.
Na próxima seção, sera apresentado o estado da arte, onde sera revisado a literatura sobre detecção e prevenção de intrusão utilizando RNA.

Logo apos teremos a proposta  de forma mais detalhada do que é pretendido realizar na próxima etapa do trabalho.

Por fim um cronograma de controle sobre como prosseguira a segunda etapa deste trabalho.


% ----------------------------------------------------------
% Capitulo 2
% ----------------------------------------------------------
\chapter[Revisão de Literatura]{Revisão de Literatura}
%\addcontentsline{toc}{chapter}{Revisão de Literatura}
% ----------------------------------------------------------


% ----------------------------------------------------------
% Capitulo 3
% ----------------------------------------------------------
\chapter[Proposta do Trabalho]{Proposta do Trabalho (O que vai ser desenvolvido!)}
%\addcontentsline{toc}{chapter}{Metodologia}
% ----------------------------------------------------------


% ----------------------------------------------------------
% Capitulo 4
% ----------------------------------------------------------
\chapter[Expectativas]{Expectativas}
%\addcontentsline{toc}{chapter}{Expectativas}
% ---



% ----------------------------------------------------------
% ELEMENTOS PÓS-TEXTUAIS
% ----------------------------------------------------------
\postextual
% ----------------------------------------------------------

% ----------------------------------------------------------
% Referências bibliográficas
% ----------------------------------------------------------
\chapter[Bibliografia]{Bibliografia}
\bibliography{abntex2-modelo-references}
%\bibliography{monografia}

[1] CERT/CC. CERT/CC Statistics 1988-2015. Centro de Estudos, Resposta e Tratamento de Incidentes de Segurança  (Coordenation Center), Maio 2015. Acessado em 22/05/2015. Disponível
em: <http://www.cert.br/stats/incidentes/2014-jan-dec/analise.html>.

[2] Miroslav Stampar, “Artificial Inteligence in network intrusion detection” Information Systems Security Bureal, 2014

[3] Jake Ryan, Meng-Jang Lin and Risto Miikkulainen. Intrusion Detection with Neural
Networks. In Advances in Neural Information Processing Systems 10, MIT Press, 1998.


[4] C. Zhang, J. Jiang and M. Kamel, “Intrusion
Detection using hierarchical neural
networks,” Pattern Recognition Letters, pp.
779-791, 2005.

[5] X. Tong, Z. Wang and H. Yu, “A research
using hybrid RBF/ Elman neural networks for
intrusion detection system secure model,”
Computer Physics Communication, pp. 1795-
1801, 2009.

[6] S.-C. O. K. Y. Wonil Kim, “Intrusion
Detection Based on Feature Transform Using
Neural Network,” in Computational Science -
ICCS 2004, vol. 3037, Springer Berlin
Heidelberg, 2004, pp. 212-219

[7] R. Beghdad, “Critical study of neural networks in detecting intrusions,” Computers \& Security, pp. 168-175, 2008.


\end{document}
